\documentclass[11pt]{article}

    \usepackage[breakable]{tcolorbox}
    \usepackage{parskip} % Stop auto-indenting (to mimic markdown behaviour)
    

    % Basic figure setup, for now with no caption control since it's done
    % automatically by Pandoc (which extracts ![](path) syntax from Markdown).
    \usepackage{graphicx}
    % Maintain compatibility with old templates. Remove in nbconvert 6.0
    \let\Oldincludegraphics\includegraphics
    % Ensure that by default, figures have no caption (until we provide a
    % proper Figure object with a Caption API and a way to capture that
    % in the conversion process - todo).
    \usepackage{caption}
    \DeclareCaptionFormat{nocaption}{}
    \captionsetup{format=nocaption,aboveskip=0pt,belowskip=0pt}

    \usepackage{float}
    \floatplacement{figure}{H} % forces figures to be placed at the correct location
    \usepackage{xcolor} % Allow colors to be defined
    \usepackage{enumerate} % Needed for markdown enumerations to work
    \usepackage{geometry} % Used to adjust the document margins
    \usepackage{amsmath} % Equations
    \usepackage{amssymb} % Equations
    \usepackage{textcomp} % defines textquotesingle
    % Hack from http://tex.stackexchange.com/a/47451/13684:
    \AtBeginDocument{%
        \def\PYZsq{\textquotesingle}% Upright quotes in Pygmentized code
    }
    \usepackage{upquote} % Upright quotes for verbatim code
    \usepackage{eurosym} % defines \euro

    \usepackage{iftex}
    \ifPDFTeX
        \usepackage[T1]{fontenc}
        \IfFileExists{alphabeta.sty}{
              \usepackage{alphabeta}
          }{
              \usepackage[mathletters]{ucs}
              \usepackage[utf8x]{inputenc}
          }
    \else
        \usepackage{fontspec}
        \usepackage{unicode-math}
    \fi

    \usepackage{fancyvrb} % verbatim replacement that allows latex
    \usepackage{grffile} % extends the file name processing of package graphics 
                         % to support a larger range
    \makeatletter % fix for old versions of grffile with XeLaTeX
    \@ifpackagelater{grffile}{2019/11/01}
    {
      % Do nothing on new versions
    }
    {
      \def\Gread@@xetex#1{%
        \IfFileExists{"\Gin@base".bb}%
        {\Gread@eps{\Gin@base.bb}}%
        {\Gread@@xetex@aux#1}%
      }
    }
    \makeatother
    \usepackage[Export]{adjustbox} % Used to constrain images to a maximum size
    \adjustboxset{max size={0.9\linewidth}{0.9\paperheight}}

    % The hyperref package gives us a pdf with properly built
    % internal navigation ('pdf bookmarks' for the table of contents,
    % internal cross-reference links, web links for URLs, etc.)
    \usepackage{hyperref}
    % The default LaTeX title has an obnoxious amount of whitespace. By default,
    % titling removes some of it. It also provides customization options.
    \usepackage{titling}
    \usepackage{longtable} % longtable support required by pandoc >1.10
    \usepackage{booktabs}  % table support for pandoc > 1.12.2
    \usepackage{array}     % table support for pandoc >= 2.11.3
    \usepackage{calc}      % table minipage width calculation for pandoc >= 2.11.1
    \usepackage[inline]{enumitem} % IRkernel/repr support (it uses the enumerate* environment)
    \usepackage[normalem]{ulem} % ulem is needed to support strikethroughs (\sout)
                                % normalem makes italics be italics, not underlines
    \usepackage{mathrsfs}
    

    
    % Colors for the hyperref package
    \definecolor{urlcolor}{rgb}{0,.145,.698}
    \definecolor{linkcolor}{rgb}{.71,0.21,0.01}
    \definecolor{citecolor}{rgb}{.12,.54,.11}

    % ANSI colors
    \definecolor{ansi-black}{HTML}{3E424D}
    \definecolor{ansi-black-intense}{HTML}{282C36}
    \definecolor{ansi-red}{HTML}{E75C58}
    \definecolor{ansi-red-intense}{HTML}{B22B31}
    \definecolor{ansi-green}{HTML}{00A250}
    \definecolor{ansi-green-intense}{HTML}{007427}
    \definecolor{ansi-yellow}{HTML}{DDB62B}
    \definecolor{ansi-yellow-intense}{HTML}{B27D12}
    \definecolor{ansi-blue}{HTML}{208FFB}
    \definecolor{ansi-blue-intense}{HTML}{0065CA}
    \definecolor{ansi-magenta}{HTML}{D160C4}
    \definecolor{ansi-magenta-intense}{HTML}{A03196}
    \definecolor{ansi-cyan}{HTML}{60C6C8}
    \definecolor{ansi-cyan-intense}{HTML}{258F8F}
    \definecolor{ansi-white}{HTML}{C5C1B4}
    \definecolor{ansi-white-intense}{HTML}{A1A6B2}
    \definecolor{ansi-default-inverse-fg}{HTML}{FFFFFF}
    \definecolor{ansi-default-inverse-bg}{HTML}{000000}

    % common color for the border for error outputs.
    \definecolor{outerrorbackground}{HTML}{FFDFDF}

    % commands and environments needed by pandoc snippets
    % extracted from the output of `pandoc -s`
    \providecommand{\tightlist}{%
      \setlength{\itemsep}{0pt}\setlength{\parskip}{0pt}}
    \DefineVerbatimEnvironment{Highlighting}{Verbatim}{commandchars=\\\{\}}
    % Add ',fontsize=\small' for more characters per line
    \newenvironment{Shaded}{}{}
    \newcommand{\KeywordTok}[1]{\textcolor[rgb]{0.00,0.44,0.13}{\textbf{{#1}}}}
    \newcommand{\DataTypeTok}[1]{\textcolor[rgb]{0.56,0.13,0.00}{{#1}}}
    \newcommand{\DecValTok}[1]{\textcolor[rgb]{0.25,0.63,0.44}{{#1}}}
    \newcommand{\BaseNTok}[1]{\textcolor[rgb]{0.25,0.63,0.44}{{#1}}}
    \newcommand{\FloatTok}[1]{\textcolor[rgb]{0.25,0.63,0.44}{{#1}}}
    \newcommand{\CharTok}[1]{\textcolor[rgb]{0.25,0.44,0.63}{{#1}}}
    \newcommand{\StringTok}[1]{\textcolor[rgb]{0.25,0.44,0.63}{{#1}}}
    \newcommand{\CommentTok}[1]{\textcolor[rgb]{0.38,0.63,0.69}{\textit{{#1}}}}
    \newcommand{\OtherTok}[1]{\textcolor[rgb]{0.00,0.44,0.13}{{#1}}}
    \newcommand{\AlertTok}[1]{\textcolor[rgb]{1.00,0.00,0.00}{\textbf{{#1}}}}
    \newcommand{\FunctionTok}[1]{\textcolor[rgb]{0.02,0.16,0.49}{{#1}}}
    \newcommand{\RegionMarkerTok}[1]{{#1}}
    \newcommand{\ErrorTok}[1]{\textcolor[rgb]{1.00,0.00,0.00}{\textbf{{#1}}}}
    \newcommand{\NormalTok}[1]{{#1}}
    
    % Additional commands for more recent versions of Pandoc
    \newcommand{\ConstantTok}[1]{\textcolor[rgb]{0.53,0.00,0.00}{{#1}}}
    \newcommand{\SpecialCharTok}[1]{\textcolor[rgb]{0.25,0.44,0.63}{{#1}}}
    \newcommand{\VerbatimStringTok}[1]{\textcolor[rgb]{0.25,0.44,0.63}{{#1}}}
    \newcommand{\SpecialStringTok}[1]{\textcolor[rgb]{0.73,0.40,0.53}{{#1}}}
    \newcommand{\ImportTok}[1]{{#1}}
    \newcommand{\DocumentationTok}[1]{\textcolor[rgb]{0.73,0.13,0.13}{\textit{{#1}}}}
    \newcommand{\AnnotationTok}[1]{\textcolor[rgb]{0.38,0.63,0.69}{\textbf{\textit{{#1}}}}}
    \newcommand{\CommentVarTok}[1]{\textcolor[rgb]{0.38,0.63,0.69}{\textbf{\textit{{#1}}}}}
    \newcommand{\VariableTok}[1]{\textcolor[rgb]{0.10,0.09,0.49}{{#1}}}
    \newcommand{\ControlFlowTok}[1]{\textcolor[rgb]{0.00,0.44,0.13}{\textbf{{#1}}}}
    \newcommand{\OperatorTok}[1]{\textcolor[rgb]{0.40,0.40,0.40}{{#1}}}
    \newcommand{\BuiltInTok}[1]{{#1}}
    \newcommand{\ExtensionTok}[1]{{#1}}
    \newcommand{\PreprocessorTok}[1]{\textcolor[rgb]{0.74,0.48,0.00}{{#1}}}
    \newcommand{\AttributeTok}[1]{\textcolor[rgb]{0.49,0.56,0.16}{{#1}}}
    \newcommand{\InformationTok}[1]{\textcolor[rgb]{0.38,0.63,0.69}{\textbf{\textit{{#1}}}}}
    \newcommand{\WarningTok}[1]{\textcolor[rgb]{0.38,0.63,0.69}{\textbf{\textit{{#1}}}}}
    
    
    % Define a nice break command that doesn't care if a line doesn't already
    % exist.
    \def\br{\hspace*{\fill} \\* }
    % Math Jax compatibility definitions
    \def\gt{>}
    \def\lt{<}
    \let\Oldtex\TeX
    \let\Oldlatex\LaTeX
    \renewcommand{\TeX}{\textrm{\Oldtex}}
    \renewcommand{\LaTeX}{\textrm{\Oldlatex}}
    % Document parameters
    % Document title
    \title{S1-basespython}
    
    
    
    
    
% Pygments definitions
\makeatletter
\def\PY@reset{\let\PY@it=\relax \let\PY@bf=\relax%
    \let\PY@ul=\relax \let\PY@tc=\relax%
    \let\PY@bc=\relax \let\PY@ff=\relax}
\def\PY@tok#1{\csname PY@tok@#1\endcsname}
\def\PY@toks#1+{\ifx\relax#1\empty\else%
    \PY@tok{#1}\expandafter\PY@toks\fi}
\def\PY@do#1{\PY@bc{\PY@tc{\PY@ul{%
    \PY@it{\PY@bf{\PY@ff{#1}}}}}}}
\def\PY#1#2{\PY@reset\PY@toks#1+\relax+\PY@do{#2}}

\@namedef{PY@tok@w}{\def\PY@tc##1{\textcolor[rgb]{0.73,0.73,0.73}{##1}}}
\@namedef{PY@tok@c}{\let\PY@it=\textit\def\PY@tc##1{\textcolor[rgb]{0.24,0.48,0.48}{##1}}}
\@namedef{PY@tok@cp}{\def\PY@tc##1{\textcolor[rgb]{0.61,0.40,0.00}{##1}}}
\@namedef{PY@tok@k}{\let\PY@bf=\textbf\def\PY@tc##1{\textcolor[rgb]{0.00,0.50,0.00}{##1}}}
\@namedef{PY@tok@kp}{\def\PY@tc##1{\textcolor[rgb]{0.00,0.50,0.00}{##1}}}
\@namedef{PY@tok@kt}{\def\PY@tc##1{\textcolor[rgb]{0.69,0.00,0.25}{##1}}}
\@namedef{PY@tok@o}{\def\PY@tc##1{\textcolor[rgb]{0.40,0.40,0.40}{##1}}}
\@namedef{PY@tok@ow}{\let\PY@bf=\textbf\def\PY@tc##1{\textcolor[rgb]{0.67,0.13,1.00}{##1}}}
\@namedef{PY@tok@nb}{\def\PY@tc##1{\textcolor[rgb]{0.00,0.50,0.00}{##1}}}
\@namedef{PY@tok@nf}{\def\PY@tc##1{\textcolor[rgb]{0.00,0.00,1.00}{##1}}}
\@namedef{PY@tok@nc}{\let\PY@bf=\textbf\def\PY@tc##1{\textcolor[rgb]{0.00,0.00,1.00}{##1}}}
\@namedef{PY@tok@nn}{\let\PY@bf=\textbf\def\PY@tc##1{\textcolor[rgb]{0.00,0.00,1.00}{##1}}}
\@namedef{PY@tok@ne}{\let\PY@bf=\textbf\def\PY@tc##1{\textcolor[rgb]{0.80,0.25,0.22}{##1}}}
\@namedef{PY@tok@nv}{\def\PY@tc##1{\textcolor[rgb]{0.10,0.09,0.49}{##1}}}
\@namedef{PY@tok@no}{\def\PY@tc##1{\textcolor[rgb]{0.53,0.00,0.00}{##1}}}
\@namedef{PY@tok@nl}{\def\PY@tc##1{\textcolor[rgb]{0.46,0.46,0.00}{##1}}}
\@namedef{PY@tok@ni}{\let\PY@bf=\textbf\def\PY@tc##1{\textcolor[rgb]{0.44,0.44,0.44}{##1}}}
\@namedef{PY@tok@na}{\def\PY@tc##1{\textcolor[rgb]{0.41,0.47,0.13}{##1}}}
\@namedef{PY@tok@nt}{\let\PY@bf=\textbf\def\PY@tc##1{\textcolor[rgb]{0.00,0.50,0.00}{##1}}}
\@namedef{PY@tok@nd}{\def\PY@tc##1{\textcolor[rgb]{0.67,0.13,1.00}{##1}}}
\@namedef{PY@tok@s}{\def\PY@tc##1{\textcolor[rgb]{0.73,0.13,0.13}{##1}}}
\@namedef{PY@tok@sd}{\let\PY@it=\textit\def\PY@tc##1{\textcolor[rgb]{0.73,0.13,0.13}{##1}}}
\@namedef{PY@tok@si}{\let\PY@bf=\textbf\def\PY@tc##1{\textcolor[rgb]{0.64,0.35,0.47}{##1}}}
\@namedef{PY@tok@se}{\let\PY@bf=\textbf\def\PY@tc##1{\textcolor[rgb]{0.67,0.36,0.12}{##1}}}
\@namedef{PY@tok@sr}{\def\PY@tc##1{\textcolor[rgb]{0.64,0.35,0.47}{##1}}}
\@namedef{PY@tok@ss}{\def\PY@tc##1{\textcolor[rgb]{0.10,0.09,0.49}{##1}}}
\@namedef{PY@tok@sx}{\def\PY@tc##1{\textcolor[rgb]{0.00,0.50,0.00}{##1}}}
\@namedef{PY@tok@m}{\def\PY@tc##1{\textcolor[rgb]{0.40,0.40,0.40}{##1}}}
\@namedef{PY@tok@gh}{\let\PY@bf=\textbf\def\PY@tc##1{\textcolor[rgb]{0.00,0.00,0.50}{##1}}}
\@namedef{PY@tok@gu}{\let\PY@bf=\textbf\def\PY@tc##1{\textcolor[rgb]{0.50,0.00,0.50}{##1}}}
\@namedef{PY@tok@gd}{\def\PY@tc##1{\textcolor[rgb]{0.63,0.00,0.00}{##1}}}
\@namedef{PY@tok@gi}{\def\PY@tc##1{\textcolor[rgb]{0.00,0.52,0.00}{##1}}}
\@namedef{PY@tok@gr}{\def\PY@tc##1{\textcolor[rgb]{0.89,0.00,0.00}{##1}}}
\@namedef{PY@tok@ge}{\let\PY@it=\textit}
\@namedef{PY@tok@gs}{\let\PY@bf=\textbf}
\@namedef{PY@tok@gp}{\let\PY@bf=\textbf\def\PY@tc##1{\textcolor[rgb]{0.00,0.00,0.50}{##1}}}
\@namedef{PY@tok@go}{\def\PY@tc##1{\textcolor[rgb]{0.44,0.44,0.44}{##1}}}
\@namedef{PY@tok@gt}{\def\PY@tc##1{\textcolor[rgb]{0.00,0.27,0.87}{##1}}}
\@namedef{PY@tok@err}{\def\PY@bc##1{{\setlength{\fboxsep}{\string -\fboxrule}\fcolorbox[rgb]{1.00,0.00,0.00}{1,1,1}{\strut ##1}}}}
\@namedef{PY@tok@kc}{\let\PY@bf=\textbf\def\PY@tc##1{\textcolor[rgb]{0.00,0.50,0.00}{##1}}}
\@namedef{PY@tok@kd}{\let\PY@bf=\textbf\def\PY@tc##1{\textcolor[rgb]{0.00,0.50,0.00}{##1}}}
\@namedef{PY@tok@kn}{\let\PY@bf=\textbf\def\PY@tc##1{\textcolor[rgb]{0.00,0.50,0.00}{##1}}}
\@namedef{PY@tok@kr}{\let\PY@bf=\textbf\def\PY@tc##1{\textcolor[rgb]{0.00,0.50,0.00}{##1}}}
\@namedef{PY@tok@bp}{\def\PY@tc##1{\textcolor[rgb]{0.00,0.50,0.00}{##1}}}
\@namedef{PY@tok@fm}{\def\PY@tc##1{\textcolor[rgb]{0.00,0.00,1.00}{##1}}}
\@namedef{PY@tok@vc}{\def\PY@tc##1{\textcolor[rgb]{0.10,0.09,0.49}{##1}}}
\@namedef{PY@tok@vg}{\def\PY@tc##1{\textcolor[rgb]{0.10,0.09,0.49}{##1}}}
\@namedef{PY@tok@vi}{\def\PY@tc##1{\textcolor[rgb]{0.10,0.09,0.49}{##1}}}
\@namedef{PY@tok@vm}{\def\PY@tc##1{\textcolor[rgb]{0.10,0.09,0.49}{##1}}}
\@namedef{PY@tok@sa}{\def\PY@tc##1{\textcolor[rgb]{0.73,0.13,0.13}{##1}}}
\@namedef{PY@tok@sb}{\def\PY@tc##1{\textcolor[rgb]{0.73,0.13,0.13}{##1}}}
\@namedef{PY@tok@sc}{\def\PY@tc##1{\textcolor[rgb]{0.73,0.13,0.13}{##1}}}
\@namedef{PY@tok@dl}{\def\PY@tc##1{\textcolor[rgb]{0.73,0.13,0.13}{##1}}}
\@namedef{PY@tok@s2}{\def\PY@tc##1{\textcolor[rgb]{0.73,0.13,0.13}{##1}}}
\@namedef{PY@tok@sh}{\def\PY@tc##1{\textcolor[rgb]{0.73,0.13,0.13}{##1}}}
\@namedef{PY@tok@s1}{\def\PY@tc##1{\textcolor[rgb]{0.73,0.13,0.13}{##1}}}
\@namedef{PY@tok@mb}{\def\PY@tc##1{\textcolor[rgb]{0.40,0.40,0.40}{##1}}}
\@namedef{PY@tok@mf}{\def\PY@tc##1{\textcolor[rgb]{0.40,0.40,0.40}{##1}}}
\@namedef{PY@tok@mh}{\def\PY@tc##1{\textcolor[rgb]{0.40,0.40,0.40}{##1}}}
\@namedef{PY@tok@mi}{\def\PY@tc##1{\textcolor[rgb]{0.40,0.40,0.40}{##1}}}
\@namedef{PY@tok@il}{\def\PY@tc##1{\textcolor[rgb]{0.40,0.40,0.40}{##1}}}
\@namedef{PY@tok@mo}{\def\PY@tc##1{\textcolor[rgb]{0.40,0.40,0.40}{##1}}}
\@namedef{PY@tok@ch}{\let\PY@it=\textit\def\PY@tc##1{\textcolor[rgb]{0.24,0.48,0.48}{##1}}}
\@namedef{PY@tok@cm}{\let\PY@it=\textit\def\PY@tc##1{\textcolor[rgb]{0.24,0.48,0.48}{##1}}}
\@namedef{PY@tok@cpf}{\let\PY@it=\textit\def\PY@tc##1{\textcolor[rgb]{0.24,0.48,0.48}{##1}}}
\@namedef{PY@tok@c1}{\let\PY@it=\textit\def\PY@tc##1{\textcolor[rgb]{0.24,0.48,0.48}{##1}}}
\@namedef{PY@tok@cs}{\let\PY@it=\textit\def\PY@tc##1{\textcolor[rgb]{0.24,0.48,0.48}{##1}}}

\def\PYZbs{\char`\\}
\def\PYZus{\char`\_}
\def\PYZob{\char`\{}
\def\PYZcb{\char`\}}
\def\PYZca{\char`\^}
\def\PYZam{\char`\&}
\def\PYZlt{\char`\<}
\def\PYZgt{\char`\>}
\def\PYZsh{\char`\#}
\def\PYZpc{\char`\%}
\def\PYZdl{\char`\$}
\def\PYZhy{\char`\-}
\def\PYZsq{\char`\'}
\def\PYZdq{\char`\"}
\def\PYZti{\char`\~}
% for compatibility with earlier versions
\def\PYZat{@}
\def\PYZlb{[}
\def\PYZrb{]}
\makeatother


    % For linebreaks inside Verbatim environment from package fancyvrb. 
    \makeatletter
        \newbox\Wrappedcontinuationbox 
        \newbox\Wrappedvisiblespacebox 
        \newcommand*\Wrappedvisiblespace {\textcolor{red}{\textvisiblespace}} 
        \newcommand*\Wrappedcontinuationsymbol {\textcolor{red}{\llap{\tiny$\m@th\hookrightarrow$}}} 
        \newcommand*\Wrappedcontinuationindent {3ex } 
        \newcommand*\Wrappedafterbreak {\kern\Wrappedcontinuationindent\copy\Wrappedcontinuationbox} 
        % Take advantage of the already applied Pygments mark-up to insert 
        % potential linebreaks for TeX processing. 
        %        {, <, #, %, $, ' and ": go to next line. 
        %        _, }, ^, &, >, - and ~: stay at end of broken line. 
        % Use of \textquotesingle for straight quote. 
        \newcommand*\Wrappedbreaksatspecials {% 
            \def\PYGZus{\discretionary{\char`\_}{\Wrappedafterbreak}{\char`\_}}% 
            \def\PYGZob{\discretionary{}{\Wrappedafterbreak\char`\{}{\char`\{}}% 
            \def\PYGZcb{\discretionary{\char`\}}{\Wrappedafterbreak}{\char`\}}}% 
            \def\PYGZca{\discretionary{\char`\^}{\Wrappedafterbreak}{\char`\^}}% 
            \def\PYGZam{\discretionary{\char`\&}{\Wrappedafterbreak}{\char`\&}}% 
            \def\PYGZlt{\discretionary{}{\Wrappedafterbreak\char`\<}{\char`\<}}% 
            \def\PYGZgt{\discretionary{\char`\>}{\Wrappedafterbreak}{\char`\>}}% 
            \def\PYGZsh{\discretionary{}{\Wrappedafterbreak\char`\#}{\char`\#}}% 
            \def\PYGZpc{\discretionary{}{\Wrappedafterbreak\char`\%}{\char`\%}}% 
            \def\PYGZdl{\discretionary{}{\Wrappedafterbreak\char`\$}{\char`\$}}% 
            \def\PYGZhy{\discretionary{\char`\-}{\Wrappedafterbreak}{\char`\-}}% 
            \def\PYGZsq{\discretionary{}{\Wrappedafterbreak\textquotesingle}{\textquotesingle}}% 
            \def\PYGZdq{\discretionary{}{\Wrappedafterbreak\char`\"}{\char`\"}}% 
            \def\PYGZti{\discretionary{\char`\~}{\Wrappedafterbreak}{\char`\~}}% 
        } 
        % Some characters . , ; ? ! / are not pygmentized. 
        % This macro makes them "active" and they will insert potential linebreaks 
        \newcommand*\Wrappedbreaksatpunct {% 
            \lccode`\~`\.\lowercase{\def~}{\discretionary{\hbox{\char`\.}}{\Wrappedafterbreak}{\hbox{\char`\.}}}% 
            \lccode`\~`\,\lowercase{\def~}{\discretionary{\hbox{\char`\,}}{\Wrappedafterbreak}{\hbox{\char`\,}}}% 
            \lccode`\~`\;\lowercase{\def~}{\discretionary{\hbox{\char`\;}}{\Wrappedafterbreak}{\hbox{\char`\;}}}% 
            \lccode`\~`\:\lowercase{\def~}{\discretionary{\hbox{\char`\:}}{\Wrappedafterbreak}{\hbox{\char`\:}}}% 
            \lccode`\~`\?\lowercase{\def~}{\discretionary{\hbox{\char`\?}}{\Wrappedafterbreak}{\hbox{\char`\?}}}% 
            \lccode`\~`\!\lowercase{\def~}{\discretionary{\hbox{\char`\!}}{\Wrappedafterbreak}{\hbox{\char`\!}}}% 
            \lccode`\~`\/\lowercase{\def~}{\discretionary{\hbox{\char`\/}}{\Wrappedafterbreak}{\hbox{\char`\/}}}% 
            \catcode`\.\active
            \catcode`\,\active 
            \catcode`\;\active
            \catcode`\:\active
            \catcode`\?\active
            \catcode`\!\active
            \catcode`\/\active 
            \lccode`\~`\~ 	
        }
    \makeatother

    \let\OriginalVerbatim=\Verbatim
    \makeatletter
    \renewcommand{\Verbatim}[1][1]{%
        %\parskip\z@skip
        \sbox\Wrappedcontinuationbox {\Wrappedcontinuationsymbol}%
        \sbox\Wrappedvisiblespacebox {\FV@SetupFont\Wrappedvisiblespace}%
        \def\FancyVerbFormatLine ##1{\hsize\linewidth
            \vtop{\raggedright\hyphenpenalty\z@\exhyphenpenalty\z@
                \doublehyphendemerits\z@\finalhyphendemerits\z@
                \strut ##1\strut}%
        }%
        % If the linebreak is at a space, the latter will be displayed as visible
        % space at end of first line, and a continuation symbol starts next line.
        % Stretch/shrink are however usually zero for typewriter font.
        \def\FV@Space {%
            \nobreak\hskip\z@ plus\fontdimen3\font minus\fontdimen4\font
            \discretionary{\copy\Wrappedvisiblespacebox}{\Wrappedafterbreak}
            {\kern\fontdimen2\font}%
        }%
        
        % Allow breaks at special characters using \PYG... macros.
        \Wrappedbreaksatspecials
        % Breaks at punctuation characters . , ; ? ! and / need catcode=\active 	
        \OriginalVerbatim[#1,codes*=\Wrappedbreaksatpunct]%
    }
    \makeatother

    % Exact colors from NB
    \definecolor{incolor}{HTML}{303F9F}
    \definecolor{outcolor}{HTML}{D84315}
    \definecolor{cellborder}{HTML}{CFCFCF}
    \definecolor{cellbackground}{HTML}{F7F7F7}
    
    % prompt
    \makeatletter
    \newcommand{\boxspacing}{\kern\kvtcb@left@rule\kern\kvtcb@boxsep}
    \makeatother
    \newcommand{\prompt}[4]{
        {\ttfamily\llap{{\color{#2}[#3]:\hspace{3pt}#4}}\vspace{-\baselineskip}}
    }
    

    
    % Prevent overflowing lines due to hard-to-break entities
    \sloppy 
    % Setup hyperref package
    \hypersetup{
      breaklinks=true,  % so long urls are correctly broken across lines
      colorlinks=true,
      urlcolor=urlcolor,
      linkcolor=linkcolor,
      citecolor=citecolor,
      }
    % Slightly bigger margins than the latex defaults
    
    \geometry{verbose,tmargin=1in,bmargin=1in,lmargin=1in,rmargin=1in}
    
    

\begin{document}
    
    \maketitle
    
    

    
    {Le langage Python}

\begin{figure}
\centering
%\includegraphics{python.png}
\caption{logo}
\end{figure}

    \begin{quote}
Python est un langage de programmation, développé en \textbf{1989} par
\textbf{Guido Von Rossum}, à l'université d'Amsterdam. Il est simple
d'usage, concis, libre et gratuit, multiplateforme, largement répandu,
riche de bibliothèques adaptées et bénéficiant d'une vaste communauté
d'auteurs dans le monde éducatif.
\end{quote}

Il permet d'interagir avec la machine à l'aide d'un programme appelé
\emph{interprète Python}. On peut l'utiliser de deux façons différentes.
La première méthode consiste en un \emph{dialogue avec l'interprète}.
c'est le mode \textbf{interactif}.

Le second consiste à \emph{écrire un programme ou code source dans un
fichier}, c'est à dire une suite d'instructions, puis à le faire
exécuter par l'interprète Python. C'est le \textbf{mode programme}.

On peut utiliser différents environnements de développement : Idle,
Pyzo, Spyder, Pyscripter\ldots, qui permettent d'utiliser les deux modes
simultanément.

Mais dans un premier temps, vous allez le faire directement dans ce
\textbf{notebook}, au fur et à mesure du cours.

    \hypertarget{sommaire}{%
\subsection{Sommaire}\label{sommaire}}

\begin{quote}
\hyperref[op]{1. Les opérateurs}
\end{quote}

\begin{quote}
\hyperref[com]{2. Commandes de bases}
\end{quote}

\begin{quote}
\hyperref[cond]{3. Les conditions}
\end{quote}

\begin{quote}
\hyperref[bou]{4. Les boucles}
\end{quote}

\begin{quote}
\hyperref[fonc]{5. Les fonctions}
\end{quote}

\begin{quote}
\hyperref[ch]{6. Les chaînes de caractères}
\end{quote}

\begin{quote}
\hyperref[boo]{7. Les booléens}
\end{quote}

    \begin{tcolorbox}[breakable, size=fbox, boxrule=1pt, pad at break*=1mm,colback=cellbackground, colframe=cellborder]
\prompt{In}{incolor}{1}{\boxspacing}
\begin{Verbatim}[commandchars=\\\{\}]
\PY{k+kn}{from} \PY{n+nn}{IPython}\PY{n+nn}{.}\PY{n+nn}{display} \PY{k+kn}{import} \PY{n}{Latex}
\end{Verbatim}
\end{tcolorbox}

    \hypertarget{les-opuxe9rateurs}{%
\subsection{1. Les opérateurs}\label{les-opuxe9rateurs}}

    On peut utilser l'interpréteur Python comme une simple machine à écrire
:

    \begin{tcolorbox}[breakable, size=fbox, boxrule=1pt, pad at break*=1mm,colback=cellbackground, colframe=cellborder]
\prompt{In}{incolor}{2}{\boxspacing}
\begin{Verbatim}[commandchars=\\\{\}]
\PY{l+m+mi}{5} \PY{o}{+} \PY{l+m+mi}{3}
\end{Verbatim}
\end{tcolorbox}

            \begin{tcolorbox}[breakable, size=fbox, boxrule=.5pt, pad at break*=1mm, opacityfill=0]
\prompt{Out}{outcolor}{2}{\boxspacing}
\begin{Verbatim}[commandchars=\\\{\}]
8
\end{Verbatim}
\end{tcolorbox}
        
    \begin{tcolorbox}[breakable, size=fbox, boxrule=1pt, pad at break*=1mm,colback=cellbackground, colframe=cellborder]
\prompt{In}{incolor}{3}{\boxspacing}
\begin{Verbatim}[commandchars=\\\{\}]
\PY{l+m+mi}{7} \PY{o}{\PYZhy{}} \PY{l+m+mi}{8}
\end{Verbatim}
\end{tcolorbox}

            \begin{tcolorbox}[breakable, size=fbox, boxrule=.5pt, pad at break*=1mm, opacityfill=0]
\prompt{Out}{outcolor}{3}{\boxspacing}
\begin{Verbatim}[commandchars=\\\{\}]
-1
\end{Verbatim}
\end{tcolorbox}
        
    \begin{tcolorbox}[breakable, size=fbox, boxrule=1pt, pad at break*=1mm,colback=cellbackground, colframe=cellborder]
\prompt{In}{incolor}{4}{\boxspacing}
\begin{Verbatim}[commandchars=\\\{\}]
\PY{l+m+mi}{16} \PY{o}{/} \PY{l+m+mi}{5}
\end{Verbatim}
\end{tcolorbox}

            \begin{tcolorbox}[breakable, size=fbox, boxrule=.5pt, pad at break*=1mm, opacityfill=0]
\prompt{Out}{outcolor}{4}{\boxspacing}
\begin{Verbatim}[commandchars=\\\{\}]
3.2
\end{Verbatim}
\end{tcolorbox}
        
    Maintenant certaines opérations un peu moins classiques sont à connaître
:

    \begin{tcolorbox}[breakable, size=fbox, boxrule=1pt, pad at break*=1mm,colback=cellbackground, colframe=cellborder]
\prompt{In}{incolor}{5}{\boxspacing}
\begin{Verbatim}[commandchars=\\\{\}]
\PY{l+m+mi}{5} \PY{o}{*}\PY{o}{*} \PY{l+m+mi}{3}
\end{Verbatim}
\end{tcolorbox}

            \begin{tcolorbox}[breakable, size=fbox, boxrule=.5pt, pad at break*=1mm, opacityfill=0]
\prompt{Out}{outcolor}{5}{\boxspacing}
\begin{Verbatim}[commandchars=\\\{\}]
125
\end{Verbatim}
\end{tcolorbox}
        
    \begin{tcolorbox}[breakable, size=fbox, boxrule=1pt, pad at break*=1mm,colback=cellbackground, colframe=cellborder]
\prompt{In}{incolor}{6}{\boxspacing}
\begin{Verbatim}[commandchars=\\\{\}]
\PY{l+m+mi}{16} \PY{o}{/}\PY{o}{/} \PY{l+m+mi}{5}
\end{Verbatim}
\end{tcolorbox}

            \begin{tcolorbox}[breakable, size=fbox, boxrule=.5pt, pad at break*=1mm, opacityfill=0]
\prompt{Out}{outcolor}{6}{\boxspacing}
\begin{Verbatim}[commandchars=\\\{\}]
3
\end{Verbatim}
\end{tcolorbox}
        
    \begin{tcolorbox}[breakable, size=fbox, boxrule=1pt, pad at break*=1mm,colback=cellbackground, colframe=cellborder]
\prompt{In}{incolor}{7}{\boxspacing}
\begin{Verbatim}[commandchars=\\\{\}]
\PY{l+m+mi}{16} \PY{o}{\PYZpc{}} \PY{l+m+mi}{5}
\end{Verbatim}
\end{tcolorbox}

            \begin{tcolorbox}[breakable, size=fbox, boxrule=.5pt, pad at break*=1mm, opacityfill=0]
\prompt{Out}{outcolor}{7}{\boxspacing}
\begin{Verbatim}[commandchars=\\\{\}]
1
\end{Verbatim}
\end{tcolorbox}
        
    Exercice 1 :

Précisez ce que font les opérateurs :

**

//

\(\%\)

    \begin{tcolorbox}[breakable, size=fbox, boxrule=1pt, pad at break*=1mm,colback=cellbackground, colframe=cellborder]
\prompt{In}{incolor}{ }{\boxspacing}
\begin{Verbatim}[commandchars=\\\{\}]
\PY{n}{Réponse} \PY{p}{:}
\end{Verbatim}
\end{tcolorbox}

    \textbf{Opérateurs de comparaison} - x == y : test d'égalité - x != y :
différent - x \textgreater{} y : supérieur - x \textgreater= y :
supérieur ou égal - x \textless{} y : inférieur - x \textless= y :
inférieur ou égal

    \begin{tcolorbox}[breakable, size=fbox, boxrule=1pt, pad at break*=1mm,colback=cellbackground, colframe=cellborder]
\prompt{In}{incolor}{9}{\boxspacing}
\begin{Verbatim}[commandchars=\\\{\}]
\PY{n}{x} \PY{p}{,} \PY{n}{y} \PY{o}{=} \PY{l+m+mi}{3} \PY{p}{,} \PY{l+m+mi}{2}
\PY{n}{x} \PY{o}{==} \PY{n}{y}
\end{Verbatim}
\end{tcolorbox}

            \begin{tcolorbox}[breakable, size=fbox, boxrule=.5pt, pad at break*=1mm, opacityfill=0]
\prompt{Out}{outcolor}{9}{\boxspacing}
\begin{Verbatim}[commandchars=\\\{\}]
False
\end{Verbatim}
\end{tcolorbox}
        
    \begin{tcolorbox}[breakable, size=fbox, boxrule=1pt, pad at break*=1mm,colback=cellbackground, colframe=cellborder]
\prompt{In}{incolor}{10}{\boxspacing}
\begin{Verbatim}[commandchars=\\\{\}]
\PY{n}{x} \PY{o}{!=} \PY{n}{y}
\end{Verbatim}
\end{tcolorbox}

            \begin{tcolorbox}[breakable, size=fbox, boxrule=.5pt, pad at break*=1mm, opacityfill=0]
\prompt{Out}{outcolor}{10}{\boxspacing}
\begin{Verbatim}[commandchars=\\\{\}]
True
\end{Verbatim}
\end{tcolorbox}
        
    \hypertarget{commandes-de-base}{%
\subsection{2. Commandes de base}\label{commandes-de-base}}

    Afficher un message de bienvenue avec \textbf{print()}.

    \begin{tcolorbox}[breakable, size=fbox, boxrule=1pt, pad at break*=1mm,colback=cellbackground, colframe=cellborder]
\prompt{In}{incolor}{ }{\boxspacing}
\begin{Verbatim}[commandchars=\\\{\}]
\PY{n+nb}{print}\PY{p}{(}\PY{l+s+s2}{\PYZdq{}}\PY{l+s+s2}{Hello world !}\PY{l+s+s2}{\PYZdq{}}\PY{p}{)}
\end{Verbatim}
\end{tcolorbox}

    Remarque

Nous sommes dans un interpréteur de code, il est inutile de d'utiliser
\textbf{print()}.

    \begin{tcolorbox}[breakable, size=fbox, boxrule=1pt, pad at break*=1mm,colback=cellbackground, colframe=cellborder]
\prompt{In}{incolor}{ }{\boxspacing}
\begin{Verbatim}[commandchars=\\\{\}]
\PY{l+s+s2}{\PYZdq{}}\PY{l+s+s2}{Hello world !}\PY{l+s+s2}{\PYZdq{}}
\end{Verbatim}
\end{tcolorbox}

    On demande une donnée (valeur numérique, texte) à l'utilisateur avec
\textbf{input()}.

    \begin{tcolorbox}[breakable, size=fbox, boxrule=1pt, pad at break*=1mm,colback=cellbackground, colframe=cellborder]
\prompt{In}{incolor}{ }{\boxspacing}
\begin{Verbatim}[commandchars=\\\{\}]
\PY{n+nb}{input}\PY{p}{(}\PY{l+s+s2}{\PYZdq{}}\PY{l+s+s2}{quel est ton prénom ?}\PY{l+s+s2}{\PYZdq{}}\PY{p}{)}
\end{Verbatim}
\end{tcolorbox}

    Remarque

L'ordinateur affiche votre réponse, mais ne la garde pas en mémoire !
Impossible d'utiliser cette réponse par la suite\ldots{}

C'est pourquoi on utilise des \textbf{variables} dans les programmes.

On leur donne un nom, mais pour l'ordinateur il s'agit d'une référence
désignant une \textbf{adresse mémoire}, c'est à dire un emplacement
précis de la mémoire. A cet emplacement est stocké une valeur bien
déterminée.

Pour \textbf{affecter} une valeur à une variable, on utilise le symbole
\textbf{=}.

    \begin{tcolorbox}[breakable, size=fbox, boxrule=1pt, pad at break*=1mm,colback=cellbackground, colframe=cellborder]
\prompt{In}{incolor}{ }{\boxspacing}
\begin{Verbatim}[commandchars=\\\{\}]
\PY{n}{var} \PY{o}{=} \PY{n+nb}{input}\PY{p}{(}\PY{l+s+s2}{\PYZdq{}}\PY{l+s+s2}{Quel est ton nom ?}\PY{l+s+s2}{\PYZdq{}}\PY{p}{)}
\end{Verbatim}
\end{tcolorbox}

    Maintenant, on peut récupérer la réponse donnée en appelant la variable
dans laquelle on l'a stockée :

    \begin{tcolorbox}[breakable, size=fbox, boxrule=1pt, pad at break*=1mm,colback=cellbackground, colframe=cellborder]
\prompt{In}{incolor}{ }{\boxspacing}
\begin{Verbatim}[commandchars=\\\{\}]
\PY{n}{var}
\end{Verbatim}
\end{tcolorbox}

    \begin{tcolorbox}[breakable, size=fbox, boxrule=1pt, pad at break*=1mm,colback=cellbackground, colframe=cellborder]
\prompt{In}{incolor}{ }{\boxspacing}
\begin{Verbatim}[commandchars=\\\{\}]
\PY{n+nb}{print}\PY{p}{(}\PY{l+s+s2}{\PYZdq{}}\PY{l+s+s2}{Bonjour }\PY{l+s+s2}{\PYZdq{}}\PY{p}{,}\PY{n}{var}\PY{p}{)}
\end{Verbatim}
\end{tcolorbox}

    Remarque : .

Pour afficher un texte, on l'écrit entre guillemets '' ``, mais pour
afficher une variable on écrit simplement son nom. On peut les afficher
simultanément en les séparant par une virgule.

Un nom de variable respecte la syntaxe suivante: - commence par une
lettre - ne contient que des lettres (sans accents), des chiffres et le
caractère \_.

Python est sensible à la casse (c'est à dire l'emploi des majuscules et
des minuscules) et a des mots clé réservés.

Par convention on nomme les variables avec des lettres minuscules, mais
on peut ajouter une majuscule pour une meilleure lisibilité.

Par exemple : \textbf{monPrenom}

    On peut combiner plusieurs instructions :

    \begin{tcolorbox}[breakable, size=fbox, boxrule=1pt, pad at break*=1mm,colback=cellbackground, colframe=cellborder]
\prompt{In}{incolor}{ }{\boxspacing}
\begin{Verbatim}[commandchars=\\\{\}]
\PY{n+nb}{print}\PY{p}{(}\PY{l+s+s2}{\PYZdq{}}\PY{l+s+s2}{8+4 =}\PY{l+s+s2}{\PYZdq{}}\PY{p}{,}\PY{l+m+mi}{8}\PY{o}{+}\PY{l+m+mi}{4}\PY{p}{)}
\end{Verbatim}
\end{tcolorbox}

    \hypertarget{les-variables}{%
\subsubsection{Les variables}\label{les-variables}}

Il en existe différents types, par exemple :

\begin{itemize}
\item
  les nombres entiers, sont les \textbf{integer}, notés \textbf{int}
\item
  les nombres décimaux, sont les \textbf{float}.
\item
  les chaînes de caractères constituées de caractères alphabétiques, de
  mots, de phrases ou de suites de symboles quelconques, sont les
  \textbf{strings}, notées \textbf{str} (Ce qui est noté entre
  guillemets simples ou doubles (voir triples) est automatiquement de
  type str.)
\item
  les booléens, notés \emph{bool}, qui ne prennent que deux valeurs :
  True ou False. Ils permettent de tester si une expression logique est
  vraie ou fausse.
\end{itemize}

Pour connaitre le type d'une variable, il suffit de taper
\textbf{type(variable)}.

    \begin{tcolorbox}[breakable, size=fbox, boxrule=1pt, pad at break*=1mm,colback=cellbackground, colframe=cellborder]
\prompt{In}{incolor}{ }{\boxspacing}
\begin{Verbatim}[commandchars=\\\{\}]
\PY{n}{a} \PY{o}{=} \PY{l+m+mf}{2.5}   \PY{c+c1}{\PYZsh{}on utilise le . pour noter les nombres décimaux}
\PY{n+nb}{type}\PY{p}{(}\PY{n}{a}\PY{p}{)}
\end{Verbatim}
\end{tcolorbox}

    \begin{tcolorbox}[breakable, size=fbox, boxrule=1pt, pad at break*=1mm,colback=cellbackground, colframe=cellborder]
\prompt{In}{incolor}{ }{\boxspacing}
\begin{Verbatim}[commandchars=\\\{\}]
\PY{n}{b} \PY{o}{=} \PY{l+m+mi}{10}
\PY{n}{a} \PY{o}{\PYZlt{}} \PY{n}{b}    \PY{c+c1}{\PYZsh{}le test est vrai, python renvoie un booléen}
\end{Verbatim}
\end{tcolorbox}

    On peut changer le type d'une variable :

    \begin{tcolorbox}[breakable, size=fbox, boxrule=1pt, pad at break*=1mm,colback=cellbackground, colframe=cellborder]
\prompt{In}{incolor}{ }{\boxspacing}
\begin{Verbatim}[commandchars=\\\{\}]
\PY{n}{c} \PY{o}{=} \PY{l+m+mi}{2}
\PY{n}{d} \PY{o}{=} \PY{n+nb}{float}\PY{p}{(}\PY{l+m+mi}{2}\PY{p}{)}  \PY{c+c1}{\PYZsh{}d est un nombre réel}
\PY{n}{d}
\end{Verbatim}
\end{tcolorbox}

    \begin{tcolorbox}[breakable, size=fbox, boxrule=1pt, pad at break*=1mm,colback=cellbackground, colframe=cellborder]
\prompt{In}{incolor}{ }{\boxspacing}
\begin{Verbatim}[commandchars=\\\{\}]
\PY{n}{e} \PY{o}{=} \PY{l+s+s1}{\PYZsq{}}\PY{l+s+s1}{2}\PY{l+s+s1}{\PYZsq{}}      \PY{c+c1}{\PYZsh{}e est une chaîne de caractères}
\PY{n+nb}{print}\PY{p}{(}\PY{n+nb}{type}\PY{p}{(}\PY{n}{e}\PY{p}{)}\PY{p}{)}
\PY{n}{e} \PY{o}{=} \PY{n+nb}{int}\PY{p}{(}\PY{n}{e}\PY{p}{)}  \PY{c+c1}{\PYZsh{}e devient un entier}
\PY{n+nb}{print}\PY{p}{(}\PY{n+nb}{type}\PY{p}{(}\PY{n}{e}\PY{p}{)}\PY{p}{)}
\end{Verbatim}
\end{tcolorbox}

    Pour tester si deux variables sont égales ou non, on utlise \textbf{==}
(à ne pas confondre avec l'affectation) ou \textbf{!=}

La réponse retournée est sous la forme d'un booléen.

    \begin{tcolorbox}[breakable, size=fbox, boxrule=1pt, pad at break*=1mm,colback=cellbackground, colframe=cellborder]
\prompt{In}{incolor}{ }{\boxspacing}
\begin{Verbatim}[commandchars=\\\{\}]
\PY{n+nb}{print}\PY{p}{(}\PY{n}{a} \PY{o}{==} \PY{l+m+mi}{2}\PY{p}{)}   \PY{c+c1}{\PYZsh{}est\PYZhy{}ce que a est égal à 2 ?}
\PY{n+nb}{print}\PY{p}{(}\PY{n}{b} \PY{o}{==} \PY{k+kc}{False}\PY{p}{)}
\PY{n+nb}{print}\PY{p}{(}\PY{n}{a} \PY{o}{!=} \PY{l+m+mi}{5}\PY{p}{)}   \PY{c+c1}{\PYZsh{}est\PYZhy{}ce que a est différent de 5 ?}
\PY{n+nb}{print}\PY{p}{(}\PY{n}{b} \PY{o}{!=} \PY{l+m+mi}{0}\PY{p}{)}
\end{Verbatim}
\end{tcolorbox}

    Remarque sur les commentaires de code

Il est très important de commenter vos programmes, pour que d'autres
puissent les comprendre et que vous-même puissiez y revenir plus tard
sans être perdu.

Pour cela on utilise le symbole \textbf{\#}. Tout ce qui suit ce
caractère ne sera pas exécuté, c'est un commentaire.

    Exercice 2 :

Ecrire ci-dessous un programme demandant deux nombres entiers et
affichant la somme de ces nombres.

    \begin{tcolorbox}[breakable, size=fbox, boxrule=1pt, pad at break*=1mm,colback=cellbackground, colframe=cellborder]
\prompt{In}{incolor}{ }{\boxspacing}
\begin{Verbatim}[commandchars=\\\{\}]
\PY{c+c1}{\PYZsh{}code}
\end{Verbatim}
\end{tcolorbox}

    Remarque

La commande input() donne une \textbf{chaîne de caractères}, même si
l'on entre un nombre.

** Ne pas confondre le caractère 4 avec le nombre 4 suivnat l'usage,
Python c'est faire la différence !

Il vaut mieux utiliser eval(input()) qui reconnaît si la variable donnée
est un nombre ou une chaîne de caractères.

    \begin{tcolorbox}[breakable, size=fbox, boxrule=1pt, pad at break*=1mm,colback=cellbackground, colframe=cellborder]
\prompt{In}{incolor}{ }{\boxspacing}
\begin{Verbatim}[commandchars=\\\{\}]
\PY{n}{a} \PY{o}{=} \PY{n+nb}{eval}\PY{p}{(}\PY{n+nb}{input}\PY{p}{(}\PY{l+s+s2}{\PYZdq{}}\PY{l+s+s2}{Donner un nombre :}\PY{l+s+s2}{\PYZdq{}}\PY{p}{)}\PY{p}{)}
\PY{n}{b} \PY{o}{=} \PY{n+nb}{eval}\PY{p}{(}\PY{n+nb}{input}\PY{p}{(}\PY{l+s+s2}{\PYZdq{}}\PY{l+s+s2}{Donner un nombre :}\PY{l+s+s2}{\PYZdq{}}\PY{p}{)}\PY{p}{)}
\PY{n}{a}\PY{o}{+}\PY{n}{b}
\end{Verbatim}
\end{tcolorbox}

    On peut faire des affectations multiples :

    \begin{tcolorbox}[breakable, size=fbox, boxrule=1pt, pad at break*=1mm,colback=cellbackground, colframe=cellborder]
\prompt{In}{incolor}{ }{\boxspacing}
\begin{Verbatim}[commandchars=\\\{\}]
\PY{n}{x} \PY{o}{=} \PY{n}{y} \PY{o}{=} \PY{l+m+mi}{2}
\PY{n+nb}{print}\PY{p}{(}\PY{l+s+s2}{\PYZdq{}}\PY{l+s+s2}{x = }\PY{l+s+s2}{\PYZdq{}}\PY{p}{,}\PY{n}{x}\PY{p}{,}\PY{l+s+s2}{\PYZdq{}}\PY{l+s+s2}{ y = }\PY{l+s+s2}{\PYZdq{}}\PY{p}{,}\PY{n}{y}\PY{p}{)}
\end{Verbatim}
\end{tcolorbox}

    On peut affecter des valeurs à plusieurs variables en même temps :

    \begin{tcolorbox}[breakable, size=fbox, boxrule=1pt, pad at break*=1mm,colback=cellbackground, colframe=cellborder]
\prompt{In}{incolor}{ }{\boxspacing}
\begin{Verbatim}[commandchars=\\\{\}]
\PY{n}{a}\PY{p}{,}\PY{n}{b} \PY{o}{=} \PY{l+m+mi}{3}\PY{p}{,}\PY{l+m+mi}{5}
\PY{n+nb}{print}\PY{p}{(}\PY{l+s+s2}{\PYZdq{}}\PY{l+s+s2}{a =}\PY{l+s+s2}{\PYZdq{}}\PY{p}{,}\PY{n}{a}\PY{p}{,}\PY{l+s+s2}{\PYZdq{}}\PY{l+s+s2}{ et b = }\PY{l+s+s2}{\PYZdq{}}\PY{p}{,}\PY{n}{b}\PY{p}{)}
\end{Verbatim}
\end{tcolorbox}

    Exercice 3

\begin{verbatim}
Compléter le code ci-dessous pour échanger les valeurs des variables a et b, puis les afficher.
\end{verbatim}

    \begin{tcolorbox}[breakable, size=fbox, boxrule=1pt, pad at break*=1mm,colback=cellbackground, colframe=cellborder]
\prompt{In}{incolor}{ }{\boxspacing}
\begin{Verbatim}[commandchars=\\\{\}]
\PY{n}{a} \PY{o}{=} \PY{l+m+mi}{10}
\PY{n}{b} \PY{o}{=} \PY{l+m+mi}{15}
\PY{c+c1}{\PYZsh{}suite du code}
\end{Verbatim}
\end{tcolorbox}

    Il peut être nécessaire d'\textbf{incrémenter} une variable (l'augmenter
d'une certaine valeur) ou de la décrémenter (la diminuer d'une certaine
valeur).

    \begin{tcolorbox}[breakable, size=fbox, boxrule=1pt, pad at break*=1mm,colback=cellbackground, colframe=cellborder]
\prompt{In}{incolor}{ }{\boxspacing}
\begin{Verbatim}[commandchars=\\\{\}]
\PY{n}{nb} \PY{o}{=} \PY{l+m+mi}{5}
\PY{n}{nb} \PY{o}{=} \PY{n}{nb} \PY{o}{+} \PY{l+m+mi}{1}   \PY{c+c1}{\PYZsh{}on augmente nb de 1}
\PY{n+nb}{print}\PY{p}{(}\PY{n}{nb}\PY{p}{)}
\PY{n}{nb} \PY{o}{+}\PY{o}{=} \PY{l+m+mi}{1}       \PY{c+c1}{\PYZsh{}même chose}
\PY{n+nb}{print}\PY{p}{(}\PY{n}{nb}\PY{p}{)}
\PY{n}{nb} \PY{o}{\PYZhy{}}\PY{o}{=} \PY{l+m+mi}{2}       \PY{c+c1}{\PYZsh{}on diminue nb de 2}
\PY{n+nb}{print}\PY{p}{(}\PY{n}{nb}\PY{p}{)}
\end{Verbatim}
\end{tcolorbox}

    Remarques

\(\bullet\) Les instructions d'un programme s'exécutent dans l'ordre où
elles sont écrites !

\(\bullet\) Lorsque l'on fait de la programmation on commet des erreurs
(des \textbf{bugs}), et l'ensemble des techniques pour les détecter et
les corriger s'appelle \textbf{debug}.

Il y a trois types d'erreurs :

\(-\) les \textbf{erreurs de syntaxe} : les règles du langage n'ont pas
été respectées (oublie d'une parenthèse, erreur de frappe\ldots) et
produisent un arrêt de fonctionnement ;

\(-\) les \textbf{erreurs sémantiques} : le programme s'exécute
parfaitement, pas de message d'erreur, mais le résultat n'est pas celui
que l'on attend ;

\(-\) les \textbf{erreurs à l'exécution} : le programme fonctionne, mais
des circonstances particulières (par exemple unfichier déplacé\ldots) se
présentent et une erreur se produit.

\begin{quote}
Il faudra sans cesse modifier et corriger vos programmes ! Pour vous y
aider, l'interpréteur Python affiche des messages d'erreurs, précisant
le type de l'erreur et la ligne où elle s'est produite.
\end{quote}

    \hypertarget{les-modules}{%
\subsubsection{Les modules}\label{les-modules}}

Beaucoup de programmes ont déjà été écrit sous Python. On en a regroupé
certains dans des fichiers appelés \textbf{modules} ou
\textbf{bibliothèques}.

Certains sont installés en même temps que Python. Mais on peut ensuite
en utiliser d'autres, suivant nos besoins. Il suffit de les ``appeler''
en début de programme.

Par exemple, il existe un module de mathématiques (math) contenant des
fonctions comme cosinus et sinus, des nombres tels que pi\ldots{}

Pour utiliser ces modules, on tape la commande : from le-nom-du-module
import*

    \begin{tcolorbox}[breakable, size=fbox, boxrule=1pt, pad at break*=1mm,colback=cellbackground, colframe=cellborder]
\prompt{In}{incolor}{ }{\boxspacing}
\begin{Verbatim}[commandchars=\\\{\}]
\PY{k+kn}{from} \PY{n+nn}{math} \PY{k+kn}{import}\PY{o}{*}
\PY{n+nb}{print}\PY{p}{(}\PY{n}{pi}\PY{p}{)}   \PY{c+c1}{\PYZsh{}on affiche une valeur approchée de pi}
\PY{n+nb}{print}\PY{p}{(}\PY{n}{sqrt}\PY{p}{(}\PY{l+m+mi}{25}\PY{p}{)}\PY{p}{)}  \PY{c+c1}{\PYZsh{}on obtient la racine carré de 25}
\end{Verbatim}
\end{tcolorbox}

    Un autre module très utile est \textbf{random}, car il permet de générer
des nombres aléatoires.

    \begin{tcolorbox}[breakable, size=fbox, boxrule=1pt, pad at break*=1mm,colback=cellbackground, colframe=cellborder]
\prompt{In}{incolor}{ }{\boxspacing}
\begin{Verbatim}[commandchars=\\\{\}]
\PY{k+kn}{from} \PY{n+nn}{random} \PY{k+kn}{import}\PY{o}{*}
\PY{n}{randint}\PY{p}{(}\PY{l+m+mi}{1}\PY{p}{,}\PY{l+m+mi}{50}\PY{p}{)}   \PY{c+c1}{\PYZsh{}obtenir un nombre entier au hasard entre 1 et 50}
\end{Verbatim}
\end{tcolorbox}

    \hypertarget{les-conditions}{%
\subsection{3. Les conditions}\label{les-conditions}}

Pour tester une certaine condition et modifier le comportement du
programme en conséquence, on utilise l'instruction \textbf{if} (si).

Exercice 4

Tester le programme ci-dessous avec différentes valeurs :

    \begin{tcolorbox}[breakable, size=fbox, boxrule=1pt, pad at break*=1mm,colback=cellbackground, colframe=cellborder]
\prompt{In}{incolor}{ }{\boxspacing}
\begin{Verbatim}[commandchars=\\\{\}]
\PY{n}{nombre} \PY{o}{=} \PY{n+nb}{eval}\PY{p}{(}\PY{n+nb}{input}\PY{p}{(}\PY{l+s+s2}{\PYZdq{}}\PY{l+s+s2}{Donner un nombre :}\PY{l+s+s2}{\PYZdq{}}\PY{p}{)}\PY{p}{)}
\PY{k}{if} \PY{p}{(}\PY{n}{nombre} \PY{o}{\PYZlt{}} \PY{l+m+mi}{0}\PY{p}{)}\PY{p}{:}
    \PY{n+nb}{print}\PY{p}{(}\PY{l+s+s2}{\PYZdq{}}\PY{l+s+s2}{Le nombre est négatif.}\PY{l+s+s2}{\PYZdq{}}\PY{p}{)}
\end{Verbatim}
\end{tcolorbox}

    Si la condition indiquée est vraie, ici le nombre est négatif, alors on
exécute l'instruction d'affichage.

On peut rajouter une instruction pour le cas où la condition est fausse,
voir rajouter d'autres cas.

    \begin{tcolorbox}[breakable, size=fbox, boxrule=1pt, pad at break*=1mm,colback=cellbackground, colframe=cellborder]
\prompt{In}{incolor}{ }{\boxspacing}
\begin{Verbatim}[commandchars=\\\{\}]
\PY{n}{nombre} \PY{o}{=} \PY{n+nb}{eval}\PY{p}{(}\PY{n+nb}{input}\PY{p}{(}\PY{l+s+s2}{\PYZdq{}}\PY{l+s+s2}{Donner un nombre :}\PY{l+s+s2}{\PYZdq{}}\PY{p}{)}\PY{p}{)}
\PY{k}{if} \PY{p}{(}\PY{n}{nombre} \PY{o}{\PYZlt{}} \PY{l+m+mi}{0}\PY{p}{)}\PY{p}{:}
    \PY{n+nb}{print}\PY{p}{(}\PY{l+s+s2}{\PYZdq{}}\PY{l+s+s2}{Le nombre est négatif.}\PY{l+s+s2}{\PYZdq{}}\PY{p}{)}
\PY{k}{elif} \PY{p}{(}\PY{n}{nombre} \PY{o}{\PYZgt{}} \PY{l+m+mi}{0}\PY{p}{)}\PY{p}{:}
    \PY{n+nb}{print}\PY{p}{(}\PY{l+s+s2}{\PYZdq{}}\PY{l+s+s2}{Le nombre est positif}\PY{l+s+s2}{\PYZdq{}}\PY{p}{)}
\PY{k}{else}\PY{p}{:}
    \PY{n+nb}{print}\PY{p}{(}\PY{l+s+s2}{\PYZdq{}}\PY{l+s+s2}{Le nombre est nul.}\PY{l+s+s2}{\PYZdq{}}\PY{p}{)}
\end{Verbatim}
\end{tcolorbox}

    Remarque importante

Le décalage à droite se nomme l'\textbf{indentation}. Il est obligatoire
pour signifier que les instructions suivantes sont dans la condition.

Les parenthèses autour de la condition ne sont pas obligatoires.

    Exercice 5

Ecrire un programme qui permet de tester si un nombre donné est pair ou
impair, et qui l'affiche.

    \begin{tcolorbox}[breakable, size=fbox, boxrule=1pt, pad at break*=1mm,colback=cellbackground, colframe=cellborder]
\prompt{In}{incolor}{ }{\boxspacing}
\begin{Verbatim}[commandchars=\\\{\}]
\PY{c+c1}{\PYZsh{}code}
\end{Verbatim}
\end{tcolorbox}

    \hypertarget{les-boucles}{%
\subsection{4. Les boucles}\label{les-boucles}}

    L'une des tâches que les machines font le mieux est la répétition de
tâches identiques.

Pour cela on peut utiliser une boucle \textbf{while} (tant que).

\emph{Tant que la condition indiquée est vraie}, on répète le bloc
d'instructions de la boucle, que l'on repère grâce à l'indentation des
lignes. Si la \emph{condition est fausse}, le bloc d'instruction est
ignoré.

    \begin{tcolorbox}[breakable, size=fbox, boxrule=1pt, pad at break*=1mm,colback=cellbackground, colframe=cellborder]
\prompt{In}{incolor}{ }{\boxspacing}
\begin{Verbatim}[commandchars=\\\{\}]
\PY{n}{z} \PY{o}{=} \PY{l+m+mi}{0}
\PY{k}{while} \PY{p}{(}\PY{n}{z} \PY{o}{\PYZlt{}} \PY{l+m+mi}{7}\PY{p}{)}\PY{p}{:}
    \PY{n}{z} \PY{o}{+}\PY{o}{=} \PY{l+m+mi}{1}
    \PY{n+nb}{print}\PY{p}{(}\PY{n}{z}\PY{p}{,}\PY{n}{end} \PY{o}{=} \PY{l+s+s2}{\PYZdq{}}\PY{l+s+s2}{ }\PY{l+s+s2}{\PYZdq{}}\PY{p}{)}
\end{Verbatim}
\end{tcolorbox}

    Remarque

L'instruction \textbf{end = '' ``} signifie que l'on veut afficher les
nombres sur la même ligne, séparés d'un espace.

La variable évaluée dans la condition doit exister au préalable.

Si la condition reste \emph{toujours vraie}, alors le corps de la boucle
est répété indéfiniment (jusqu'à ce que Python cesse de fonctionner). Il
faut absoluement l'éviter !

    Exercice 6

Ecrire un programme qui affiche les 20 premiers termes de la table de
multiplication de 8.

    \begin{tcolorbox}[breakable, size=fbox, boxrule=1pt, pad at break*=1mm,colback=cellbackground, colframe=cellborder]
\prompt{In}{incolor}{ }{\boxspacing}
\begin{Verbatim}[commandchars=\\\{\}]
\PY{c+c1}{\PYZsh{}code}
\end{Verbatim}
\end{tcolorbox}

    Exercice 7

On lance deux dés à 6 faces parfaitement équilibrés et on additionne les
deux résultats obtenus.

Ecrire un programme qui simule cette expérience et indique le nombre de
lancés nécessaires pour que la somme soit égale à 12.

    \begin{tcolorbox}[breakable, size=fbox, boxrule=1pt, pad at break*=1mm,colback=cellbackground, colframe=cellborder]
\prompt{In}{incolor}{ }{\boxspacing}
\begin{Verbatim}[commandchars=\\\{\}]
\PY{c+c1}{\PYZsh{}code}
\end{Verbatim}
\end{tcolorbox}

    Une autre boucle, la boucle \textbf{for} (pour).

On répète un certain nombre de fois le bloc d'instructions

    \begin{tcolorbox}[breakable, size=fbox, boxrule=1pt, pad at break*=1mm,colback=cellbackground, colframe=cellborder]
\prompt{In}{incolor}{ }{\boxspacing}
\begin{Verbatim}[commandchars=\\\{\}]
\PY{k}{for} \PY{n}{i} \PY{o+ow}{in} \PY{n+nb}{range}\PY{p}{(}\PY{l+m+mi}{10}\PY{p}{)}\PY{p}{:}
    \PY{n+nb}{print}\PY{p}{(}\PY{n}{i}\PY{p}{,}\PY{n}{end} \PY{o}{=} \PY{l+s+s2}{\PYZdq{}}\PY{l+s+s2}{;}\PY{l+s+s2}{\PYZdq{}}\PY{p}{)}
\end{Verbatim}
\end{tcolorbox}

    Remarque

\textbf{i} est ce qu'on appelle un \textbf{compteur}, qui prendra
successivement les valeurs entières de 0 à 9.

Et à chaque fois, on exécuter le bloc d'instructions de la boucle.

    Quelques variantes :

    \begin{tcolorbox}[breakable, size=fbox, boxrule=1pt, pad at break*=1mm,colback=cellbackground, colframe=cellborder]
\prompt{In}{incolor}{ }{\boxspacing}
\begin{Verbatim}[commandchars=\\\{\}]
\PY{k}{for} \PY{n}{i} \PY{o+ow}{in} \PY{n+nb}{range}\PY{p}{(}\PY{l+m+mi}{1}\PY{p}{,}\PY{l+m+mi}{10}\PY{p}{)}\PY{p}{:}
    \PY{n+nb}{print}\PY{p}{(}\PY{n}{i}\PY{p}{,}\PY{n}{end} \PY{o}{=} \PY{l+s+s2}{\PYZdq{}}\PY{l+s+s2}{;}\PY{l+s+s2}{\PYZdq{}}\PY{p}{)}
\end{Verbatim}
\end{tcolorbox}

    \begin{tcolorbox}[breakable, size=fbox, boxrule=1pt, pad at break*=1mm,colback=cellbackground, colframe=cellborder]
\prompt{In}{incolor}{ }{\boxspacing}
\begin{Verbatim}[commandchars=\\\{\}]
\PY{k}{for} \PY{n}{i} \PY{o+ow}{in} \PY{n+nb}{range}\PY{p}{(}\PY{l+m+mi}{1}\PY{p}{,}\PY{l+m+mi}{20}\PY{p}{,}\PY{l+m+mi}{2}\PY{p}{)}\PY{p}{:}
    \PY{n+nb}{print}\PY{p}{(}\PY{n}{i}\PY{p}{,}\PY{n}{end} \PY{o}{=} \PY{l+s+s2}{\PYZdq{}}\PY{l+s+s2}{;}\PY{l+s+s2}{\PYZdq{}}\PY{p}{)}
\end{Verbatim}
\end{tcolorbox}

    Exercice 8

Ecrire un programme qui calcule la somme des 25 premiers nombres
entiers.

    \begin{tcolorbox}[breakable, size=fbox, boxrule=1pt, pad at break*=1mm,colback=cellbackground, colframe=cellborder]
\prompt{In}{incolor}{ }{\boxspacing}
\begin{Verbatim}[commandchars=\\\{\}]
\PY{c+c1}{\PYZsh{}code}
\end{Verbatim}
\end{tcolorbox}

    Exercice 9

Ecrire un programme faisant deviner un nombre entier compris entre 1 et
100.

Ce nombre sera choisi au hasard par l'ordinateur ; on indiquera si le
nombre proposé est trop grand ou trop petit, ainsi que le nombre de
propositions faites jusque-là.

Si l'on dépasse les 6 propositions, on arrête le jeu et on affiche Game
Over.

    \begin{tcolorbox}[breakable, size=fbox, boxrule=1pt, pad at break*=1mm,colback=cellbackground, colframe=cellborder]
\prompt{In}{incolor}{ }{\boxspacing}
\begin{Verbatim}[commandchars=\\\{\}]
\PY{c+c1}{\PYZsh{}code}
\end{Verbatim}
\end{tcolorbox}

    \hypertarget{les-fonctions}{%
\subsection{5. Les fonctions}\label{les-fonctions}}

    Remarque importante Les mots suivants ne peuvent être utilisés comme nom
de variables, ni comme noms de fonctions :

\begin{quote}
and ; as ; assert ; break ; class ; continue ; def ; del ; elif ; else ;
except ; False ; finally ; for ; from ; global ; if ; import ; in ; is ;
lambda ; None ; nonlocal ; not ; or ; pass ; raise ; return ; True ; try
; while ; with ; yield
\end{quote}

    On a déjà utilisé certaines \textbf{fonctions} préprogrammées, comme
\textbf{print()}, \textbf{input()}, certaines sont regroupées dans des
\emph{modules}\ldots{} ; on peut également en écrire de nouvelles.

Ce sont des suites d'instructions que l'on isolent du reste du
programme, auxquelles on donne un nom, ce qui permet d'appeler la
fonction par ce nom à n'importe quel endroit du programme.

On peut ainsi utiliser ces fonctions sans avoir à les réécrire, et
autant de fois qu'on le souhaite. Elles permettent de rendre le code du
programme plus court et plus facile à comprendre.

La syntaxe d'une fonction :

\textbf{def nom\_fonction(paramètres):}

\begin{quote}
Bloc d'instructions
\end{quote}

    Remarques

\begin{itemize}
\item
  il faut indenter les instructions de la fonction
\item
  on peut donner un ou plusieurs paramètres nécessaires à l'exécution
  des instructions
\item
  une fonction ne fait rien tant qu'elle n'a pas été appelée ; on
  l'appelle par son nom.
\end{itemize}

    \begin{tcolorbox}[breakable, size=fbox, boxrule=1pt, pad at break*=1mm,colback=cellbackground, colframe=cellborder]
\prompt{In}{incolor}{ }{\boxspacing}
\begin{Verbatim}[commandchars=\\\{\}]
\PY{k}{def} \PY{n+nf}{table\PYZus{}de\PYZus{}7}\PY{p}{(}\PY{p}{)}\PY{p}{:}              \PY{c+c1}{\PYZsh{}une fonction sans paramètre}
    \PY{k}{for} \PY{n}{i} \PY{o+ow}{in} \PY{n+nb}{range}\PY{p}{(}\PY{l+m+mi}{11}\PY{p}{)}\PY{p}{:}
        \PY{n+nb}{print}\PY{p}{(}\PY{l+s+s2}{\PYZdq{}}\PY{l+s+s2}{7*}\PY{l+s+s2}{\PYZdq{}}\PY{p}{,}\PY{n}{i}\PY{p}{,}\PY{l+s+s2}{\PYZdq{}}\PY{l+s+s2}{=}\PY{l+s+s2}{\PYZdq{}}\PY{p}{,}\PY{l+m+mi}{7}\PY{o}{*}\PY{n}{i}\PY{p}{,}\PY{n}{end} \PY{o}{=}\PY{l+s+s2}{\PYZdq{}}\PY{l+s+s2}{ ; }\PY{l+s+s2}{\PYZdq{}}\PY{p}{)}
\end{Verbatim}
\end{tcolorbox}

    \begin{tcolorbox}[breakable, size=fbox, boxrule=1pt, pad at break*=1mm,colback=cellbackground, colframe=cellborder]
\prompt{In}{incolor}{ }{\boxspacing}
\begin{Verbatim}[commandchars=\\\{\}]
\PY{n}{table\PYZus{}de\PYZus{}7}\PY{p}{(}\PY{p}{)}     \PY{c+c1}{\PYZsh{}on appelle la fonction}
\end{Verbatim}
\end{tcolorbox}

    \begin{tcolorbox}[breakable, size=fbox, boxrule=1pt, pad at break*=1mm,colback=cellbackground, colframe=cellborder]
\prompt{In}{incolor}{ }{\boxspacing}
\begin{Verbatim}[commandchars=\\\{\}]
\PY{k}{def} \PY{n+nf}{table\PYZus{}mul}\PY{p}{(}\PY{n}{nb}\PY{p}{,}\PY{n+nb}{max}\PY{p}{)}\PY{p}{:}   \PY{c+c1}{\PYZsh{}on utilise deux paramètres}
    \PY{l+s+sd}{\PYZdq{}\PYZdq{}\PYZdq{}paramètres :      }
\PY{l+s+sd}{       nb : la table demandée}
\PY{l+s+sd}{       max : la valeur maximale de la table\PYZdq{}\PYZdq{}\PYZdq{}}
    \PY{k}{for} \PY{n}{i} \PY{o+ow}{in} \PY{n+nb}{range}\PY{p}{(}\PY{n+nb}{max}\PY{o}{+}\PY{l+m+mi}{1}\PY{p}{)}\PY{p}{:}
        \PY{n+nb}{print}\PY{p}{(}\PY{n}{nb}\PY{p}{,}\PY{l+s+s2}{\PYZdq{}}\PY{l+s+s2}{*}\PY{l+s+s2}{\PYZdq{}}\PY{p}{,}\PY{n}{i}\PY{p}{,}\PY{l+s+s2}{\PYZdq{}}\PY{l+s+s2}{=}\PY{l+s+s2}{\PYZdq{}}\PY{p}{,}\PY{n}{nb}\PY{o}{*}\PY{n}{i}\PY{p}{,}\PY{n}{end} \PY{o}{=}\PY{l+s+s2}{\PYZdq{}}\PY{l+s+s2}{ ; }\PY{l+s+s2}{\PYZdq{}}\PY{p}{)}    
\end{Verbatim}
\end{tcolorbox}

    \begin{tcolorbox}[breakable, size=fbox, boxrule=1pt, pad at break*=1mm,colback=cellbackground, colframe=cellborder]
\prompt{In}{incolor}{ }{\boxspacing}
\begin{Verbatim}[commandchars=\\\{\}]
\PY{n}{table\PYZus{}mul}\PY{p}{(}\PY{l+m+mi}{8}\PY{p}{,}\PY{l+m+mi}{20}\PY{p}{)}
\end{Verbatim}
\end{tcolorbox}

    Remarque

Les triples guillemets dans une fonction, permettent de faire un
commentaire où l'on précise les paramètres attendus.

    Maintenant, une fonction est créée avant tout pour \textbf{renvoyer une
valeur}, ce qui se fait avec la commande \textbf{return}.

    \begin{tcolorbox}[breakable, size=fbox, boxrule=1pt, pad at break*=1mm,colback=cellbackground, colframe=cellborder]
\prompt{In}{incolor}{ }{\boxspacing}
\begin{Verbatim}[commandchars=\\\{\}]
\PY{k}{def} \PY{n+nf}{carre}\PY{p}{(}\PY{n}{x}\PY{p}{)}\PY{p}{:}
    \PY{k}{return} \PY{n}{x}\PY{o}{*}\PY{o}{*}\PY{l+m+mi}{2}  \PY{c+c1}{\PYZsh{}cette fonction donne le carré du nombre x donné}
\end{Verbatim}
\end{tcolorbox}

    \begin{tcolorbox}[breakable, size=fbox, boxrule=1pt, pad at break*=1mm,colback=cellbackground, colframe=cellborder]
\prompt{In}{incolor}{ }{\boxspacing}
\begin{Verbatim}[commandchars=\\\{\}]
\PY{n}{carre}\PY{p}{(}\PY{l+m+mi}{5}\PY{p}{)}
\end{Verbatim}
\end{tcolorbox}

    Exercice 10

Ecrire une fonction qui permet de trouver les diviseurs d'un nombre
entier positif non nul, et qui les affiche.

    \begin{tcolorbox}[breakable, size=fbox, boxrule=1pt, pad at break*=1mm,colback=cellbackground, colframe=cellborder]
\prompt{In}{incolor}{ }{\boxspacing}
\begin{Verbatim}[commandchars=\\\{\}]
\PY{k}{def} \PY{n+nf}{diviseurs}\PY{p}{(}\PY{n}{n}\PY{p}{)}\PY{p}{:}
    \PY{l+s+sd}{\PYZdq{}\PYZdq{}\PYZdq{}Une valeur entière n non nulle est donné.}
\PY{l+s+sd}{    Cette fonction affiche l\PYZsq{}ensemble des diviseurs de n\PYZdq{}\PYZdq{}\PYZdq{}}
    \PY{k}{pass}            \PY{c+c1}{\PYZsh{}compléter le code}
\end{Verbatim}
\end{tcolorbox}

    \hypertarget{variables-locales-et-variables-globales}{%
\subsubsection{Variables locales et variables
globales}\label{variables-locales-et-variables-globales}}

    Lorsque des variables sont définies à l'intérieur du corps d'une
fonction, ces variables ne sont accessibles qu'à la fonction elle-même :
ce sont des \textbf{variables locales} ;

Lorsque des variables sont définies à l'extérieur d'une fonction : ce
sont des \textbf{variables globales}. Leur contenu est visible à
l'intérieur d'une fonction, mais la fonction ne peut pas les modifier.

    Exercice 11

Testez les exemples suivants, et commenter les résultats obtenus.

    \begin{tcolorbox}[breakable, size=fbox, boxrule=1pt, pad at break*=1mm,colback=cellbackground, colframe=cellborder]
\prompt{In}{incolor}{ }{\boxspacing}
\begin{Verbatim}[commandchars=\\\{\}]
\PY{k}{def} \PY{n+nf}{monter}\PY{p}{(}\PY{p}{)}\PY{p}{:}
    \PY{n}{v} \PY{o}{=} \PY{l+m+mi}{5}
    \PY{k}{return} \PY{n}{a}\PY{o}{*}\PY{l+m+mi}{2}

\PY{n+nb}{print}\PY{p}{(}\PY{n}{v}\PY{p}{)}  \PY{c+c1}{\PYZsh{}La variable locale v est inconnue à l\PYZsq{}extérieur de la fonction }
\end{Verbatim}
\end{tcolorbox}

    \begin{tcolorbox}[breakable, size=fbox, boxrule=1pt, pad at break*=1mm,colback=cellbackground, colframe=cellborder]
\prompt{In}{incolor}{ }{\boxspacing}
\begin{Verbatim}[commandchars=\\\{\}]
\PY{n}{m} \PY{o}{=} \PY{l+m+mi}{2}
\PY{k}{def} \PY{n+nf}{test}\PY{p}{(}\PY{p}{)}\PY{p}{:}
    \PY{k}{return} \PY{n}{m}\PY{o}{+}\PY{l+m+mi}{1}  \PY{c+c1}{\PYZsh{}on a accès à la variable globale  dans la fonction}

\PY{n}{test}\PY{p}{(}\PY{p}{)}
\end{Verbatim}
\end{tcolorbox}

    \begin{tcolorbox}[breakable, size=fbox, boxrule=1pt, pad at break*=1mm,colback=cellbackground, colframe=cellborder]
\prompt{In}{incolor}{ }{\boxspacing}
\begin{Verbatim}[commandchars=\\\{\}]
\PY{k}{def} \PY{n+nf}{test\PYZus{}2}\PY{p}{(}\PY{p}{)}\PY{p}{:}
    \PY{n}{m} \PY{o}{=} \PY{n}{m} \PY{o}{+} \PY{l+m+mi}{8}   \PY{c+c1}{\PYZsh{}on ne peut modifier la variable globale m dans la fonction}
    \PY{k}{return} \PY{n}{m}

\PY{n}{test\PYZus{}2}\PY{p}{(}\PY{p}{)}
\end{Verbatim}
\end{tcolorbox}

    \begin{tcolorbox}[breakable, size=fbox, boxrule=1pt, pad at break*=1mm,colback=cellbackground, colframe=cellborder]
\prompt{In}{incolor}{ }{\boxspacing}
\begin{Verbatim}[commandchars=\\\{\}]
\PY{k}{def} \PY{n+nf}{test\PYZus{}2}\PY{p}{(}\PY{p}{)}\PY{p}{:}
    \PY{k}{global} \PY{n}{m}    \PY{c+c1}{\PYZsh{}cette déclaration permet de modifier la variable globale dans cette fonction}
    \PY{n}{m} \PY{o}{=} \PY{n}{m} \PY{o}{+} \PY{l+m+mi}{8}   
    \PY{k}{return} \PY{n}{m}

\PY{n}{test\PYZus{}2}\PY{p}{(}\PY{p}{)}
\end{Verbatim}
\end{tcolorbox}

    \hypertarget{chauxeenes-de-caractuxe8res}{%
\subsection{6. Chaînes de
caractères}\label{chauxeenes-de-caractuxe8res}}

    Les chaînes de caractères (string) sont des types de données
\emph{composites}, car elles rassemblent dans une seule structure un
ensemble d'entités plus simples : des caractères. Ainsi on peut la
traiter comme un seul objet, ou bien comme une \emph{séquence} de
caractères distincts. On doit donc pouvoir \textbf{accéder séparément à
chacun des caractères} de la chaîne.

Pour cela on utilise la syntaxe suivante :

\textbf{nom\_de\_chaîne{[}indice{]}} où l'indice correspond à la
position du caractère dans la chaîne. (Attention, la numérotation
commence à 0)

    \begin{tcolorbox}[breakable, size=fbox, boxrule=1pt, pad at break*=1mm,colback=cellbackground, colframe=cellborder]
\prompt{In}{incolor}{ }{\boxspacing}
\begin{Verbatim}[commandchars=\\\{\}]
\PY{n}{ch} \PY{o}{=} \PY{l+s+s2}{\PYZdq{}}\PY{l+s+s2}{Bienvenue sur Mars !}\PY{l+s+s2}{\PYZdq{}}           
\PY{n+nb}{print}\PY{p}{(}\PY{n}{ch}\PY{p}{[}\PY{l+m+mi}{0}\PY{p}{]}\PY{p}{)}            \PY{c+c1}{\PYZsh{}on accède au premier caractère}
\PY{n+nb}{print}\PY{p}{(}\PY{n}{ch}\PY{p}{[}\PY{l+m+mi}{4}\PY{p}{]}\PY{p}{,}\PY{l+s+s2}{\PYZdq{}}\PY{l+s+s2}{\PYZhy{}}\PY{l+s+s2}{\PYZdq{}}\PY{p}{,}\PY{n}{ch}\PY{p}{[}\PY{l+m+mi}{9}\PY{p}{]}\PY{p}{,}\PY{l+s+s2}{\PYZdq{}}\PY{l+s+s2}{\PYZhy{}}\PY{l+s+s2}{\PYZdq{}}\PY{p}{)}  \PY{c+c1}{\PYZsh{}l\PYZsq{}espace est un caractère que l\PYZsq{}on peut afficher}
\PY{n+nb}{print}\PY{p}{(}\PY{n}{ch}\PY{p}{[}\PY{l+m+mi}{1}\PY{p}{:}\PY{l+m+mi}{4}\PY{p}{]}\PY{p}{)}          \PY{c+c1}{\PYZsh{}on affiche une partie de la chaîne : de l\PYZsq{}indice 1 à l\PYZsq{}indice 3!!}
\PY{n+nb}{print}\PY{p}{(}\PY{n}{ch}\PY{p}{[}\PY{o}{\PYZhy{}}\PY{l+m+mi}{1}\PY{p}{]}\PY{p}{)}           \PY{c+c1}{\PYZsh{}on affiche le dernier}
\PY{n+nb}{print}\PY{p}{(}\PY{n}{ch}\PY{p}{[}\PY{l+m+mi}{10}\PY{p}{:}\PY{p}{]}\PY{p}{)}           \PY{c+c1}{\PYZsh{}on affiche toute la chaîne à partir de l\PYZsq{}indice 10}
\end{Verbatim}
\end{tcolorbox}

    On peut écrire un texte long sur plusieurs lignes grâce au symbole ****
(\emph{antislash}).

Dans une chaîne de caractères, la séquence \textbf{$\backslash$n** provoque un saut
de ligne, lorsqu'on utilise la fonction }print()**.

    \begin{tcolorbox}[breakable, size=fbox, boxrule=1pt, pad at break*=1mm,colback=cellbackground, colframe=cellborder]
\prompt{In}{incolor}{ }{\boxspacing}
\begin{Verbatim}[commandchars=\\\{\}]
\PY{n}{phrase} \PY{o}{=} \PY{l+s+s2}{\PYZdq{}}\PY{l+s+s2}{Je vais essayer de ne pas }\PY{l+s+se}{\PYZbs{}}
\PY{l+s+s2}{faire de fautes d}\PY{l+s+s2}{\PYZsq{}}\PY{l+s+s2}{ortographe, }\PY{l+s+se}{\PYZbs{}n}\PY{l+s+s2}{ }\PY{l+s+se}{\PYZbs{}}
\PY{l+s+s2}{mais c}\PY{l+s+s2}{\PYZsq{}}\PY{l+s+s2}{est très difficile !}\PY{l+s+s2}{\PYZdq{}}

\PY{n+nb}{print}\PY{p}{(}\PY{n}{phrase}\PY{p}{)}
\end{Verbatim}
\end{tcolorbox}

    Il existe de nombreuses fonctions permettant de traiter les chaînes de
caractères, que l'on appelle \textbf{méthodes}.

La syntaxe pour les utiliser : \textbf{nom-chaine.methode()}

    \begin{tcolorbox}[breakable, size=fbox, boxrule=1pt, pad at break*=1mm,colback=cellbackground, colframe=cellborder]
\prompt{In}{incolor}{ }{\boxspacing}
\begin{Verbatim}[commandchars=\\\{\}]
\PY{n+nb}{print}\PY{p}{(}\PY{n}{ch}\PY{o}{.}\PY{n}{upper}\PY{p}{(}\PY{p}{)}\PY{p}{)}     \PY{c+c1}{\PYZsh{}tous les caractères s\PYZsq{}écrivent en majuscules}
\PY{n+nb}{print}\PY{p}{(}\PY{n}{ch}\PY{o}{.}\PY{n}{lower}\PY{p}{(}\PY{p}{)}\PY{p}{)}     \PY{c+c1}{\PYZsh{}tous les caractères s\PYZsq{}écrivent en minuscules}
\PY{n+nb}{print}\PY{p}{(}\PY{n}{ch}\PY{o}{.}\PY{n}{count}\PY{p}{(}\PY{l+s+s1}{\PYZsq{}}\PY{l+s+s1}{e}\PY{l+s+s1}{\PYZsq{}}\PY{p}{)}\PY{p}{)}  \PY{c+c1}{\PYZsh{}on indique le nombre de fois où le caractère e apparaît dans la chaîne}
\PY{n+nb}{print}\PY{p}{(}\PY{n}{ch}\PY{o}{.}\PY{n}{find}\PY{p}{(}\PY{l+s+s1}{\PYZsq{}}\PY{l+s+s1}{v}\PY{l+s+s1}{\PYZsq{}}\PY{p}{)}\PY{p}{)}   \PY{c+c1}{\PYZsh{}on cherche l\PYZsq{}indice de la première occurence du caractère v}
\end{Verbatim}
\end{tcolorbox}

    Parmi les fonctions les plus utiles on retrouve également
\textbf{len(nom\_chaine)} qui indique la longueur de la chaîne.

    \begin{tcolorbox}[breakable, size=fbox, boxrule=1pt, pad at break*=1mm,colback=cellbackground, colframe=cellborder]
\prompt{In}{incolor}{ }{\boxspacing}
\begin{Verbatim}[commandchars=\\\{\}]
\PY{n+nb}{len}\PY{p}{(}\PY{n}{ch}\PY{p}{)}
\end{Verbatim}
\end{tcolorbox}

    On peut choisir un caractère au hasard dans la chaîne, avec le module
\textbf{random} et la fonction \textbf{choice()} :

    \begin{tcolorbox}[breakable, size=fbox, boxrule=1pt, pad at break*=1mm,colback=cellbackground, colframe=cellborder]
\prompt{In}{incolor}{ }{\boxspacing}
\begin{Verbatim}[commandchars=\\\{\}]
\PY{k+kn}{from} \PY{n+nn}{random} \PY{k+kn}{import} \PY{n}{choice}  \PY{c+c1}{\PYZsh{}on importe seulement la fonction choice !}
\PY{n}{choice}\PY{p}{(}\PY{n}{ch}\PY{p}{)}
\end{Verbatim}
\end{tcolorbox}

    On peut assembler deux chaînes de caractères, ce qu'on appelle la
\textbf{concaténation}, en utilisant l'opérateur \textbf{+}.

    \begin{tcolorbox}[breakable, size=fbox, boxrule=1pt, pad at break*=1mm,colback=cellbackground, colframe=cellborder]
\prompt{In}{incolor}{ }{\boxspacing}
\begin{Verbatim}[commandchars=\\\{\}]
\PY{n}{ch2} \PY{o}{=} \PY{l+s+s2}{\PYZdq{}}\PY{l+s+s2}{Il y fait un peu chaud !!}\PY{l+s+s2}{\PYZdq{}}
\PY{n}{ch} \PY{o}{+} \PY{n}{ch2}
\end{Verbatim}
\end{tcolorbox}

    On peut \textbf{répéter} une partie de la chaîne :

    \begin{tcolorbox}[breakable, size=fbox, boxrule=1pt, pad at break*=1mm,colback=cellbackground, colframe=cellborder]
\prompt{In}{incolor}{ }{\boxspacing}
\begin{Verbatim}[commandchars=\\\{\}]
\PY{n}{ch}\PY{p}{[}\PY{l+m+mi}{0}\PY{p}{:}\PY{l+m+mi}{4}\PY{p}{]}\PY{o}{*}\PY{l+m+mi}{5}
\end{Verbatim}
\end{tcolorbox}

    Exercice 12

Ecrire un script qui recopie une chaîne (dans une nouvelle variable), en
insérant des astérisques entre les caractères.

    \begin{tcolorbox}[breakable, size=fbox, boxrule=1pt, pad at break*=1mm,colback=cellbackground, colframe=cellborder]
\prompt{In}{incolor}{ }{\boxspacing}
\begin{Verbatim}[commandchars=\\\{\}]
\PY{c+c1}{\PYZsh{}code de l\PYZsq{}exercice 12.}
\end{Verbatim}
\end{tcolorbox}

    Remarque

Les chaînes sont des séquences \textbf{non modifiables} de caractères.

    Exercice 13

Créer un jeu dans lequel un joueur entre un mot, l'ordinateur mélange
les lettres, et affiche le mot mélangé.

Un second joueur doit alors proposer le mot donné par le premier joueur.

\emph{Aide} : Pour mélanger les lettres on pourra commencer par mélanger
deux d'entre elles au hasard, puis répéter l'opération plusieurs fois.

    \begin{tcolorbox}[breakable, size=fbox, boxrule=1pt, pad at break*=1mm,colback=cellbackground, colframe=cellborder]
\prompt{In}{incolor}{ }{\boxspacing}
\begin{Verbatim}[commandchars=\\\{\}]
\PY{c+c1}{\PYZsh{}code        }
\end{Verbatim}
\end{tcolorbox}

    Une chaîne de caractères étant une \emph{séquence}, on peut la parcourir
avec le couple d'instructions \textbf{for \ldots{} in \ldots{}}

    \begin{tcolorbox}[breakable, size=fbox, boxrule=1pt, pad at break*=1mm,colback=cellbackground, colframe=cellborder]
\prompt{In}{incolor}{ }{\boxspacing}
\begin{Verbatim}[commandchars=\\\{\}]
\PY{n}{nom} \PY{o}{=} \PY{l+s+s2}{\PYZdq{}}\PY{l+s+s2}{Obelix}\PY{l+s+s2}{\PYZdq{}}
\PY{k}{for} \PY{n}{caractere} \PY{o+ow}{in} \PY{n}{nom}\PY{p}{:}  \PY{c+c1}{\PYZsh{}on traite successivement tous les caractères de la chaîne}
    \PY{n+nb}{print}\PY{p}{(}\PY{n}{caractere}\PY{p}{)}
\end{Verbatim}
\end{tcolorbox}

    On peut de la même manière chercher si un caractère appartient à la
chaîne.

    \begin{tcolorbox}[breakable, size=fbox, boxrule=1pt, pad at break*=1mm,colback=cellbackground, colframe=cellborder]
\prompt{In}{incolor}{ }{\boxspacing}
\begin{Verbatim}[commandchars=\\\{\}]
\PY{n}{manger} \PY{o}{=} \PY{l+s+s2}{\PYZdq{}}\PY{l+s+s2}{sanglier}\PY{l+s+s2}{\PYZdq{}}
\PY{n}{lettre} \PY{o}{=} \PY{l+s+s2}{\PYZdq{}}\PY{l+s+s2}{o}\PY{l+s+s2}{\PYZdq{}}
\PY{k}{if} \PY{n}{lettre} \PY{o+ow}{in} \PY{n}{manger}\PY{p}{:} \PY{c+c1}{\PYZsh{}on compare chaque caractère de la chaîne à la lettre proposée}
    \PY{n+nb}{print}\PY{p}{(}\PY{l+s+s2}{\PYZdq{}}\PY{l+s+s2}{Trouvé !}\PY{l+s+s2}{\PYZdq{}}\PY{p}{)}
\PY{k}{else}\PY{p}{:}
    \PY{n+nb}{print}\PY{p}{(}\PY{l+s+s2}{\PYZdq{}}\PY{l+s+s2}{Inconnu !}\PY{l+s+s2}{\PYZdq{}}\PY{p}{)}
\end{Verbatim}
\end{tcolorbox}

    \hypertarget{booluxe9ens}{%
\subsection{7. Booléens}\label{booluxe9ens}}

    Les comparaisons et tests d'égalité sont des expressions qui produisent
un résultat \textbf{booléen} : vrai (True) ou faux (False).

On peut \textbf{affecter un booléen} à une variable.

    \begin{tcolorbox}[breakable, size=fbox, boxrule=1pt, pad at break*=1mm,colback=cellbackground, colframe=cellborder]
\prompt{In}{incolor}{ }{\boxspacing}
\begin{Verbatim}[commandchars=\\\{\}]
\PY{n}{continuer} \PY{o}{=} \PY{k+kc}{True}
\PY{k}{while} \PY{n}{continuer}\PY{p}{:}              \PY{c+c1}{\PYZsh{}la condition est vraie, on exécute le bloc de la boucle while}
    \PY{n+nb}{print}\PY{p}{(}\PY{l+s+s2}{\PYZdq{}}\PY{l+s+s2}{ok}\PY{l+s+s2}{\PYZdq{}}\PY{p}{)}
    \PY{n}{poursuivre} \PY{o}{=} \PY{n+nb}{input}\PY{p}{(}\PY{l+s+s2}{\PYZdq{}}\PY{l+s+s2}{Voulez\PYZhy{}vous continuer ? o/n }\PY{l+s+s2}{\PYZdq{}}\PY{p}{)}
    \PY{k}{if} \PY{n}{poursuivre}\PY{o}{.}\PY{n}{lower}\PY{p}{(}\PY{p}{)} \PY{o}{==} \PY{l+s+s1}{\PYZsq{}}\PY{l+s+s1}{n}\PY{l+s+s1}{\PYZsq{}}\PY{p}{:}
        \PY{n}{continuer} \PY{o}{=} \PY{k+kc}{False}     \PY{c+c1}{\PYZsh{}la condition est fausse, on sort de la boucle}
\end{Verbatim}
\end{tcolorbox}

    On associe aux booléens des opérateurs permettent de combiner plusieurs
tests.

c1 et c2 sont deux conditions :

\textbf{not c1} (la négation) est réalisée lorsque la condition c1 n'est
pas réalisée ;

\textbf{c1 and c2} (la conjonction) est réalisée lorsque les conditions
c1 et c2 sont toutes deux réalisées ;

\textbf{c1 or c2} (la disjonction) est réalisée lorsqu'au moins l'une
des conditions c1 ou c2 est réalisée ;

\textbf{c1 xor c2} (la disjonction exclulsive) est réalisée lorsque
l'une des deux conditions c1 ou c2 est réalisée, mais pas l'autre ; en
langage Python, xor s'écrit à l'aide du symbole \textbf{\^{}}.

    \begin{tcolorbox}[breakable, size=fbox, boxrule=1pt, pad at break*=1mm,colback=cellbackground, colframe=cellborder]
\prompt{In}{incolor}{ }{\boxspacing}
\begin{Verbatim}[commandchars=\\\{\}]
\PY{n}{x} \PY{o}{=} \PY{l+m+mi}{5}
\PY{k}{if} \PY{p}{(}\PY{n}{x} \PY{o}{\PYZgt{}}\PY{o}{=} \PY{l+m+mi}{0}\PY{p}{)} \PY{o+ow}{and} \PY{p}{(}\PY{n}{x} \PY{o}{\PYZlt{}}\PY{o}{=} \PY{l+m+mi}{10}\PY{p}{)}\PY{p}{:}
    \PY{n+nb}{print}\PY{p}{(}\PY{l+s+s2}{\PYZdq{}}\PY{l+s+s2}{Ce nombre appartient à l}\PY{l+s+s2}{\PYZsq{}}\PY{l+s+s2}{intervalle [0 ; 10]}\PY{l+s+s2}{\PYZdq{}}\PY{p}{)}
\PY{k}{else}\PY{p}{:}
    \PY{n+nb}{print}\PY{p}{(}\PY{l+s+s2}{\PYZdq{}}\PY{l+s+s2}{Ce nombre n}\PY{l+s+s2}{\PYZsq{}}\PY{l+s+s2}{appartient pas à l}\PY{l+s+s2}{\PYZsq{}}\PY{l+s+s2}{intervalle [0 ; 10]}\PY{l+s+s2}{\PYZdq{}}\PY{p}{)}
\end{Verbatim}
\end{tcolorbox}

    Remarque

Les parenthèses permettent de mieux visualiser les tests.

    Exercice 14

Créer un jeu demandant un nombre entier compris entre 1 et 100 à un
joueur et qui lui indique quels sont ses gains, suivant les règles
suivantes :

\begin{itemize}
\item
  si le nombre est pair et supérieur ou égal à 90, il perd 1 euros ;
\item
  si le nombre est impair ou compris strictement entre 25 et 90, il ne
  gagne rien ;
\item
  sinon, il gagne 2 euros.
\end{itemize}

    \begin{tcolorbox}[breakable, size=fbox, boxrule=1pt, pad at break*=1mm,colback=cellbackground, colframe=cellborder]
\prompt{In}{incolor}{2}{\boxspacing}
\begin{Verbatim}[commandchars=\\\{\}]
\PY{c+c1}{\PYZsh{}\PYZsh{} Le code du jeu.}
\end{Verbatim}
\end{tcolorbox}

    \hyperref[som]{Retour au sommaire}


    % Add a bibliography block to the postdoc
    
    
    
\end{document}
